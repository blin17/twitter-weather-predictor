There are approximately 200 million active users on Twitter and 400 million tweets are posted on a daily basis. We can leverage these tweets about everyday topics to learn machine learning methods and keywords that best process a tweet's intent and sentiment. With that in mind, the specific domain of weather is particularly rich data set to begin our study– weather is a common topic of conversation, and this is no different in the Twitter sphere. With twitter as an individual's outlet, he can express his joy about the sun, complain about overbearing heat, or talk about the upcoming forecast. We aim to build a classifier than can extract a tweet's sentiment, time frame, and weather condition of tweets given prior data. Then we can examine our classifier to determine which algorithms are most effective and which words are the best indicators.

To answer our questions, we applied different learning methods on weather related tweets. Our findings showed that Suppport Vector Machines and Naive Bayes were accurate classifiers to extract weather information from a tweet. In addition, we discovered a few keywords in a tweet were the best indicators, while the rest of the words had a little or no effect on a tweet's sentiment, timeframe, and weather condition. Finally, we created a Markov Random Field classifier that used a tweet's relative timeframe and location to predict a tweet's message.